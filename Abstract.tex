%%xx General points
%%xx 0 Some suggestions in the text itself are shown with "xx" -- so if you just search for "xx" you will find them --> Done
%%xx 1 Make sure tenses match throughout
%%xx 2 use \citet for citations in-line (text), e.g., "\citet{monaghanpaper} found that SPH is great...", not "Monaghan \citep{monaghanpaper} found that SPH is great..." -->Done
%%xx 3 Need a section on the atmospheric structure, and how atmosphere can be incorporated into model, especially NWP or radiosondes
%%xx 4 You often seem to forget to put a space between paragraphs.  LaTeX of course does not recognize a <return> as a paragraph delimiter, nor is it proper to end a normal paragraph with \\ or % -->Done
%%xx 5 You are inconsistent with capitalization in section headings.  Either always use an initial capital letter, or only use an initial capital letter on the first word, proper names or acronyms
%%xx 6 Figures are of generally poor quality -- use PDF and reproduce at larger scale.  
%%xx 7 Difficult to understand what figure of Pinatubo plume is supposed to be telling us about the model


%% Copernicus Publications Manuscript Preparation Template for LaTeX Submissions
%% ---------------------------------
%% This template should be used for copernicus.cls
%% The class file and some style files are bundled in the Copernicus Latex Package, which can be downloaded from the different journal webpages.
%% For further assistance please contact Copernicus Publications at: production@copernicus.org
%% http://publications.copernicus.org/for_authors/manuscript_preparation.html
%% Please use the following documentclass and journal abbreviations for discussion papers and final revised papers.
%% 2-column papers and discussion papers
\documentclass[journal abbreviation, manuscript]{copernicus}
%% Journal abbreviations (please use the same for discussion papers and final revised papers)
% Archives Animal Breeding (aab)
% Atmospheric Chemistry and Physics (acp)
% Advances in Geosciences (adgeo)
% Advances in Statistical Climatology, Meteorology and Oceanography (ascmo)
% Annales Geophysicae (angeo)
% ASTRA Proceedings (ap)
% Atmospheric Measurement Techniques (amt)
% Advances in Radio Science (ars)
% Advances in Science and Research (asr)
% Biogeosciences (bg)
% Climate of the Past (cp)
% Drinking Water Engineering and Science (dwes)
% Earth System Dynamics (esd)
% Earth Surface Dynamics (esurf)
% Earth System Science Data (essd)
% Fossil Record (fr)
% Geographica Helvetica (gh)
% Geoscientific Instrumentation, Methods and Data Systems (gi)
% Geoscientific Model Development (gmd)
% Geothermal Energy Science (gtes)
% Hydrology and Earth System Sciences (hess)
% History of Geo- and Space Sciences (hgss)
% Journal of Sensors and Sensor Systems (jsss)
% Mechanical Sciences (ms)
% Natural Hazards and Earth System Sciences (nhess)
% Nonlinear Processes in Geophysics (npg)
% Ocean Science (os)
% Proceedings of the International Association of Hydrological Sciences (piahs)
% Primate Biology (pb)
% Scientific Drilling (sd)
% SOIL (soil)
% Solid Earth (se)
% The Cryosphere (tc)
% Web Ecology (we)
% Wind Energy Science (wes)
%% \usepackage commands included in the copernicus.cls:
%\usepackage[german, english]{babel}
%\usepackage{tabularx}
%\usepackage{cancel}
%\usepackage{multirow}
%\usepackage{supertabular}
%\usepackage{algorithmic}
%\usepackage{algorithm}
%\usepackage{amsthm}
%\usepackage{float}
%\usepackage{subfig}
%\usepackage{rotating}
\begin{document}
%<<<<<<< HEAD

\title{Plume-SPH 1.0: A three-dimensional, dusty-gas volcanic plume model based on smoothed particle hydrodynamics}

%>>>>>>> 0d96d9682fb484b8b159e8d09a0eb89723d3a0f8
% \Author[affil]{given_name}{surname}
\Author[1]{Zhixuan}{Cao}
\Author[1]{Abani}{Patra}
\Author[2]{Marcus}{Bursik}
\Author[3]{E. Bruce}{Pitman}
\Author[4]{Matthew}{Jones}
\affil[1]{Department of Mechanical and Aerospace Engineering, University at Buffalo, SUNY, New York, United State}
\affil[2]{Department of Geology, University at Buffalo, SUNY, New York, United State}
\affil[3]{Department of Material Design and Innovation, University at Buffalo, SUNY, New York, United State}
\affil[4]{Center for Computational Research,
University at Buffalo, SUNY, New York, United State}
%% The [] brackets identify the author with the corresponding affiliation. 1, 2, 3, etc. should be inserted.
\runningtitle{Plume-SPH 1.0}
\runningauthor{Z. Cao et al}
%<<<<<<< HEAD
\correspondence{Abani Patra (abani@buffalo.edu)}
%=======
%\correspondence{Abani Patra (abani.patra@gmail.com)}
%>>>>>>> 0d96d9682fb484b8b159e8d09a0eb89723d3a0f8
\received{}
\pubdiscuss{} %% only important for two-stage journals
\revised{}
\accepted{}
\published{}
%% These dates will be inserted by Copernicus Publications during the typesetting process.
\firstpage{1}
\maketitle
\begin{abstract}

Plume-SPH provides the the first particle based simulation of volcanic plumes.  SPH (smoothed particle hydrodynamics) has several advantages over currently used mesh based methods in modeling of multiphase free boundary flows like volcanic plumes. This tool will provide more accurate eruption source terms to users of VATDs (Volcanic ash transport and dispersion models) greatly improving volcanic ash forecasts.  The accuracy of these terms is crucial for forecasts from VATDs and the 3D SPH model presented here will provide better numerical accuracy. As an initial effort to exploit the feasibility and advantages of SPH in volcanic plume modeling, we adopt a relatively simple physics model  (3D dusty-gas dynamic model assuming well mixed eruption materiel and dynamic and thermodynamic equilibrium between air and erupted material and minimal effect of winds) targeted at capturing the salient features of a volcanic plume. The documented open source code  is easily obtained and extended to incorporate other models of physics of interest to the large community of researchers investigating multiphase free boundary flows of volcanic or other origins. 

The Plume-SPH code also incorporates  several newly developed techniques in SPH  needed to address numerical challenges in simulating multiphase compressible turbulent flow.  The code should thus be also of general interest to the much larger community of researchers using and developing SPH based tools. In particular, the $SPH-\varepsilon$ turbulence model is to capture mixing at unresolved scales, heat exchange due to turbulence is calculated by a Reynolds analogy and a corrected SPH is used to handle tensile instability and deficiency of particle distribution near the boundaries. We also developed methodology to impose velocity inlet and pressure outlet boundary conditions, both of which are scarce in traditional implementations of SPH. 

The core solver of our model is parallelized with MPI (message passing interface) obtaining good weak and strong scalability using novel techniques for data management using a SFCs (space-filling curves) and object creation time based indexing and hash table based storage scheme. These techniques are of  interest to  researchers engaged in developing particle in cell type methods. The model is verified by comparing velocity and concentration distribution along the central axis and on the transverse cross with experimental results of JPUE (jet or plume that is ejected from a nozzle into a uniform environment) and the top height of the Pinatubo eruption of 15 June 1991. Our results are consistent with both observations and existing 3D plume models. Profiles of several integrated variables are compared with those calculated in existing 3D plume models, and further verify our model. Analysis of the plume evolution process illustrates that this model is able to reproduce the physics of plume development. 

%
%
%VATDs (Volcanic ash transport and dispersion models) used for volcanic ash forecasts often take outputs of eruption plume models as source terms. The accuracy of these terms is crucial for forecasts from VATDs. All existing 3D (three dimensional) plume models described in the literature use mesh based methods. SPH (smoothed particle hydrodynamics), as a mesh free method, has several advantages over mesh based methods in modeling of multiphase free boundary flows. As an initial effort to exploit the feasibility and advantages of SPH in volcanic plume modeling, we adopt a relatively simple model, a 3D dusty-gas dynamic model, targeted at capturing the salient features of a volcanic plume. In this model, erupted material is assumed to be well-mixed, and represented by one phase while air is another phase. We also assume dynamic and thermodynamic equilibrium between air and erupted material and that the effect of wind is not significant. 
%
%Several newly developed techniques in SPH are adapted and extended to address numerical challenges in simulating multiphase compressible turbulent flow with classical SPH method. The $SPH-\varepsilon$ turbulence model is to capture mixing at unresolved scales. Heat exchange due to turbulence is then calculated by Reynolds analogy. A corrected SPH is used to handle tensile instability and deficiency of particle distribution near the boundaries. We also propose to impose velocity inlet and pressure outlet boundary conditions, both of which are scarce in traditional implementations of SPH. The core solver of our model is parallelized with MPI (message passing interface) obtaining good weak and strong scalability. The model is verified by comparing velocity and concentration distribution along the central axis and on the transverse cross with experimental results of JPUE (jet or plume that is ejected from a nozzle into a uniform environment). Next, the top height of the Pinatubo eruption of 15 June 1991, as simulated by our model, is consistent with both observations and existing 3D plume models. Profiles of several integrated variables are compared with those calculated in existing 3D plume models, and further verify our model. Analysis of the plume evolution process illustrates that this model is able to reproduce the basic physics of plume development. The comparison also implies that the turbulence model plays a significant role in 3D volcanic plume modeling.
\end{abstract}
\end{document}